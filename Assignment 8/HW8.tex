\documentclass{article}

\usepackage{fancyhdr}
\usepackage{extramarks}
\usepackage{amsmath}
\usepackage{minted}
\usepackage{amsthm}
\usepackage{amsfonts}
\usepackage{tikz}
\usepackage[plain]{algorithm}
\usepackage{algpseudocode}
\usepackage{graphicx}
\usepackage{caption}
\usepackage{subcaption}

\usetikzlibrary{automata,positioning}
\usepackage{fullpage,enumitem,amsmath,amssymb,graphicx}

%
% Basic Document Settings
%

\topmargin=-0.75in
\textwidth=6.5in
\textheight=9.0in
\headsep=0.20in
\headheight = 12pt
\linespread{1.1}

\pagestyle{fancy}
\chead{\hmwkClass\ (\hmwkClassInstructor): \hmwkTitle}
\rhead{\firstxmark}
\lfoot{\lastxmark}
\cfoot{\thepage}

\renewcommand\headrulewidth{0.55pt}
\renewcommand\footrulewidth{0.55pt}

\setlength\parindent{0pt}


\setcounter{secnumdepth}{0}

%
% Homework Problem Environment
%
% This environment takes an optional argument. When given, it will adjust the
% problem counter. This is useful for when the problems given for your
% assignment aren't sequential. See the last 3 problems of this template for an
% example.

%
% Homework Details
%   - Title
%   - Due date
%   - Class
%   - Section/Time
%   - Instructor
%   - Author
%

\newcommand{\hmwkTitle}{Homework\ \#8}
\newcommand{\hmwkDueDate}{November 23, 2021}
\newcommand{\hmwkClassCode}{COT 5615}
\newcommand{\hmwkClass}{Math for Intelligent Systems}
\newcommand{\hmwkClassYear}{Fall 2021}
\newcommand{\hmwkClassInstructor}{Professor Kejun Huang}
\newcommand{\hmwkAuthorName}{\textit{Vyom Pathak}}
\newcommand{\hmwkUFID}{96703101}

%
%
%
% Various Helper Commands
%

% Useful for algorithms
\newcommand{\alg}[1]{\textsc{\bfseries \footnotesize #1}}

% For derivatives
\newcommand{\deriv}[1]{\frac{\mathrm{d}}{\mathrm{d}x} (#1)}

% For partial derivatives
\newcommand{\pderiv}[2]{\frac{\partial}{\partial #1} (#2)}

% Integral dx
\newcommand{\dx}{\mathrm{d}x}

% Alias for the Solution section header
\newcommand{\solution}{\textbf{\large Solution}}

% Probability commands: Expectation, Variance, Covariance, Bias
\newcommand{\E}{\mathrm{E}}
\newcommand{\Var}{\mathrm{Var}}
\newcommand{\Cov}{\mathrm{Cov}}
\newcommand{\Bias}{\mathrm{Bias}}

% norm bars
\newcommand{\norm}[1]{\left\lVert#1\right\rVert}

\begin{document}

\begin{center}
{\Large \hmwkClassCode\ \hmwkClass\ \hmwkClassYear\ \hmwkTitle}

\begin{tabular}{rl}
UFID: & \hmwkUFID \\
Name: & \hmwkAuthorName \\
Instructor: & \hmwkClassInstructor \\
Due Date: & \hmwkDueDate \\ 
% Collaborators: & [list all the people you worked with]
\end{tabular}
\end{center}

\section*{Problem A16.2}
\subsection*{Julia timing test for least-norm}
\subsubsection*{Solution}
Following is the code to calculate the time take for the least-norm problem:
\begin{minted}[
frame=lines,
framesep=2mm,
baselinestretch=1.2,
fontsize=\footnotesize,
linenos
]{julia}
    
    m = 600
    n = 4000
    A = rand(m,n);
    b = rand(m);
    @time A\b;
    @time A\b;
    @time A\b;
    @time A\b;
    @time A\b;
  #1.609742 seconds (488.74 k allocations: 69.135 MiB, 11.70% compilation time)
  #1.308426 seconds (3.64 k allocations: 41.110 MiB)
  #1.295313 seconds (3.64 k allocations: 41.110 MiB, 0.97% gc time)
  #1.341974 seconds (3.64 k allocations: 41.110 MiB, 3.30% gc time)
  #1.316347 seconds (3.64 k allocations: 41.110 MiB)
    \end{minted}
The approximate time taken is  $1.312\ seconds$. The approximate flop rate is given as : $\frac{2nm^2}{average\_time} = \frac{2*4000*(600)^2}{1.312} = 2.195\ GFlops/second$.
\section*{Problem A16.3}
\subsection*{Constrained least squares in Julia}
\subsubsection*{Solution}
The following code is used to find constrained least square solution along with the Lagrangian's multiplier:
    \begin{minted}[
frame=lines,
framesep=2mm,
baselinestretch=1.2,
fontsize=\footnotesize,
linenos
]{julia}
    using LinearAlgebra
    m = 10000
    n = 1000
    p = 100
    
    A = rand(m,n);
    b = rand(m,1);
    C = rand(p,n);
    d = rand(p,1);
    KKT =  [2*A'A C';C zeros(p,p)]\[2*A'b;d];
    
    print("x hat (solution): \n");
    display(KKT[1:n,1]);
    print("\n Lagrange Multiplier: \n");
    display(KKT[n+1:n+p,1]);

    \end{minted}
\section*{Problem A16.4}
\subsection*{Varying the right-hand sides in linearly constrained least squares}
\subsubsection*{Solution}
The colleague is correct. We can find $F$ and $G$ using the formula for finding $\hat{x}$ from the QR factorization: 
\begin{align*}
    R\hat{x} & = Q^T_1 b - (1/2)Q^T_2 w\\
    R\hat{x} & = Q^T_1 b - (1/2)Q^T_2 (z-2d)\ [\because w = z-2d]\\
    R\hat{x} & = Q^T_1 b - (1/2)Q^T_2 z + Q^T_2d\\
    R\hat{x} & = Q^T_1 b + Q^T_2d\  [\because The\ columns\ of\ Q^T_2\ are\ linearly\ independent\ which\ implies\ that\ z = 0]\\
    \hat{x} & = R^{-1} Q^T_1 b + R^{-1}Q^T_2d\\
\end{align*}
Thus, for the solution eqaution $\hat{x} = Fb+Gd$, we get $F = R^{-1}Q^T_1$ and $G = R^{-1}Q^T_2$.
\section*{Problem 4}
\subsection*{LDLT Factorization for constrained least squares}
\subsubsection*{Solution}
Here, we are trying to minimize $\norm{y}^2$ with constraints $Ax + y = b$. This can also be written as $\norm{A_nX_n-b_n}^2$ with $CX_n=d$, where $X_n = (x, y)$, $C = [A_n;I]$, $A_n = (0,0;0,I)$, $b_n = (0,0)$, $d=b$.\\
Now, $(2A^T_nA_n, C;C^T,0)(X_n,z) = (0,d)$ [$\because Using\ KKT\ Matrix$].\\
This is simplified as follows: 
$(2A_n, C;C^T,0)(X_n,z) = (0,d)$.\\
From solving these tqo equations by expanding all the variables, we get the following two equations:\\
$A^Ty=0$ and $Ax+Iy=b$, which are the final two equations.\\
These equations can also be expressed as the following matrix:
$(0,A;A^T,I)(x,y) = (0,b)$.
\end{document}
