\documentclass{article}

\usepackage{fancyhdr}
\usepackage{extramarks}
\usepackage{amsmath}
\usepackage{minted}
\usepackage{amsthm}
\usepackage{amsfonts}
\usepackage{tikz}
\usepackage[plain]{algorithm}
\usepackage{algpseudocode}

\usetikzlibrary{automata,positioning}
\usepackage{fullpage,enumitem,amsmath,amssymb,graphicx}

%
% Basic Document Settings
%

\topmargin=-0.75in
\textwidth=6.5in
\textheight=9.0in
\headsep=0.20in
\headheight = 12pt
\linespread{1.1}

\pagestyle{fancy}
\chead{\hmwkClass\ (\hmwkClassInstructor): \hmwkTitle}
\rhead{\firstxmark}
\lfoot{\lastxmark}
\cfoot{\thepage}

\renewcommand\headrulewidth{0.55pt}
\renewcommand\footrulewidth{0.55pt}

\setlength\parindent{0pt}


\setcounter{secnumdepth}{0}

%
% Homework Problem Environment
%
% This environment takes an optional argument. When given, it will adjust the
% problem counter. This is useful for when the problems given for your
% assignment aren't sequential. See the last 3 problems of this template for an
% example.

%
% Homework Details
%   - Title
%   - Due date
%   - Class
%   - Section/Time
%   - Instructor
%   - Author
%

\newcommand{\hmwkTitle}{Homework\ \#1}
\newcommand{\hmwkDueDate}{September 06, 2021}
\newcommand{\hmwkClassCode}{COT 5615}
\newcommand{\hmwkClass}{Math for Intelligent Systems}
\newcommand{\hmwkClassYear}{Fall 2021}
\newcommand{\hmwkClassInstructor}{Professor Kejun Huang}
\newcommand{\hmwkAuthorName}{\textit{Vyom Pathak}}
\newcommand{\hmwkUFID}{96703101}

%
%
%
% Various Helper Commands
%

% Useful for algorithms
\newcommand{\alg}[1]{\textsc{\bfseries \footnotesize #1}}

% For derivatives
\newcommand{\deriv}[1]{\frac{\mathrm{d}}{\mathrm{d}x} (#1)}

% For partial derivatives
\newcommand{\pderiv}[2]{\frac{\partial}{\partial #1} (#2)}

% Integral dx
\newcommand{\dx}{\mathrm{d}x}

% Alias for the Solution section header
\newcommand{\solution}{\textbf{\large Solution}}

% Probability commands: Expectation, Variance, Covariance, Bias
\newcommand{\E}{\mathrm{E}}
\newcommand{\Var}{\mathrm{Var}}
\newcommand{\Cov}{\mathrm{Cov}}
\newcommand{\Bias}{\mathrm{Bias}}

\begin{document}

\begin{center}
{\Large \hmwkClassCode\ \hmwkClass\ \hmwkClassYear\ \hmwkTitle}

\begin{tabular}{rl}
UFID: & \hmwkUFID \\
Name: & \hmwkAuthorName \\
Instructor: & \hmwkClassInstructor \\
Due Date: & \hmwkDueDate \\ 
% Collaborators: & [list all the people you worked with]
\end{tabular}
\end{center}

\section*{Problem 1.7}
\subsection*{Transforming between two encodings for Boolean vectors}
\subsubsection*{Solution}
Vector $y$ in terms of $x$ can be shown as follows: $y = 2x - 1$. \\ Ex: 
when $x_i = 1$ we get $y_i = 2 \cdot 1-1 = + 1$; and when $x_i = 0$ we get $y_i = 2 \cdot 0-1 = -1$.\\
Similarly, $x$ in terms of $y$ can be shown as follows: $x = (1/2)(y + 1)$

\section*{Problem 1.13}
\subsection*{Average age in a population}
\subsubsection*{Solution}
\begin{enumerate}[label=(\alph*)]
  \item The total population can be calculated as $1^Tx$
  \item The total number of people aged 65 or over can be calculated as $a^Tx$ where $a = (0_{65},1_{35})$ 
  \item The average age of the population can be calculated as
  \begin{align*}
    % \begin{equation}
        \frac{(0, 1, 2, ... , 99)^Tx}{1^Tx}
    % \end{equation}
\end{align*}
\end{enumerate}

\section*{Problem 1.17}
\subsection*{Linear combinations of cash flows}
\subsubsection*{Solution}
  From the given details, $c$ can be written as follows, 
  \begin{align*}
      c = (1,0,\ldots,0,-(1+r)^{T-1})
  \end{align*}
Single period loan, at time period $t$ can be written as follows,
\begin{align*}
    l_t = (0,\ldots,0,1,-(1+r),0,\ldots,0),\ \  t = 1,\ldots,T-1
\end{align*}
So, solution is to take a new loan each time period to cover the amount you owe up until that period. So, after taking \$1 in period 1, we take out a loan for \$(1+r) in period 2, and end up owing \$(1+r)$^2$ in period 3. Then we take out a loan for \$(1+r)$^2$ in period 3, and so on. Therefore, We can express the $c$ vector using the above loan period definition as follows,
\begin{align*}
    (1,0,\ldots,0,-(1+r)^{T-1}) = (1,-(1+r),0,0,\ldots,0,0) + (0,1,-(1+r),0,0,\ldots,0,0)\cdot(1+r) \\  + (0,0,1,-(1+r),0,0,\ldots,0,0)\cdot(1+r)^2 + \ldots + (0,0,\ldots,0,1,-(1+r))\cdot(1+r)^{T-2}  
\end{align*}
This can be further simplified as follows,
\begin{align*}
    c = l_1 + l_2\cdot(1+r) + l_3\cdot(1+r)^2 + \ldots + l_{T-1}\cdot(1+r)^{T-2}
\end{align*}
Here, the coefficients of linear combinations are $1, 1+r, (1+r)^2, \ldots, (1+r)^{T-2}$.
\section*{Problem A1.2}
\subsection*{Creating vectors in Julia}
\subsubsection*{Solution}
\begin{enumerate}[label=(\alph*)]

  \begin{minted}[
frame=lines,
framesep=2mm,
baselinestretch=1.2,
fontsize=\footnotesize,
linenos
]{julia}
    #a)
    a = (1:10 .== 5)
    x = rand(10)
    print(transpose(a)*x,"\n")
    #b)
    a = [0.3, 0.4, 0.3]
    x = rand(3)
    print(transpose(a)*x,"\n")
    #c)
    x = rand(22)
    a = zeros(22)
    for i in 1:22
        if mod(i,4) == 0
             a[i] = 1
        elseif mod(i,7) == 0
             a[i] = -1
        end
    end
    print(transpose(a)*x,"\n")
    #d)
    x = rand(11)
    a = zeros(11)
    a[convert(Int32,((length(a)-5)//2))+1:convert(Int32,((length(a)+5)//2))] = ones(5)
    print(transpose(a)*x/5)
\end{minted}
\end{enumerate}

\section*{Problem A1.6}
\subsubsection*{Solution}
For any inner product, we need $n$ multiplications and $n-1$ additions, so in total we generate $2n-1$ flops. Thus for $10^6$-vector inner product, we generate 11 flops which are calculated in 0.001 seconds. Now, for $10^7$-vector inner product, we generate 13 flops which are thus calculated in $(13\cdot0.001)/11 = 0.0011\overline{81}$ seconds.

\section*{Problem 2.4}
\subsection*{Linear function}
\subsubsection*{Solution}
$\phi$ cannot be linear. Here, the third
point $(1, -1, -1)$ is the opposite of the second point $(-1, 1, 1)$. If $\phi$ were linear, the two values of $\phi$ would need to be opposite (negative) of each other. But they are both 1 and 1. Thus, the function is not linear.
\section*{Problem 2.8}
\subsection*{Integral and derivative of polynomial}
\subsubsection*{Solution}

\begin{enumerate}[label=(\alph*)]
\item Here the integration is calculated as follows
\begin{align*}
    \int_{\alpha}^{\beta} p(x)\ dx = c_1(\beta - \alpha) + \frac{c_2}{2}(\beta^2 - \alpha^2) + \ldots +  \frac{c_n}{n}(\beta^n - \alpha^n)
\end{align*}
Therefore,
\begin{align*}
    \alpha = (\beta - \alpha , \frac{\beta^2 - \alpha^2}{2} , \ldots ,  \frac{\beta^n - \alpha^n}{n})
\end{align*}
\item Here the derivation at $\hat x$ is calculated as follows
\begin{align*}
p'(\alpha) = c_2 + 2c_3\alpha + 3c_4\alpha^{2} +  \ldots +  (n-1)c_n\alpha^{n-2}
\end{align*}
Therefore,
\begin{align*}
b = (0,1,2\alpha,3\alpha^{2} ,  \ldots ,  (n-1)\alpha^{n-2})
\end{align*}
\end{enumerate}
\end{document}
